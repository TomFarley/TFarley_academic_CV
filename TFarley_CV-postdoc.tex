%% start of file `template.tex'.
%% Copyright 2006-2015 Xavier Danaux (xdanaux@gmail.com).
%
% This work may be distributed and/or modified under the
% conditions of the LaTeX Project Public License version 1.3c,
% available at http://www.latex-project.org/lppl/.

% Setup
% sudo apt-get install texlive-generic-ext

\documentclass[11pt,a4paper,sans]{moderncv}        % possible options include font size ('10pt', '11pt' and '12pt'), paper size ('a4paper', 'letterpaper', 'a5paper', 'legalpaper', 'executivepaper' and 'landscape') and font family ('sans' and 'roman')

% moderncv themes
\moderncvstyle{fancy}                             % style options are 'casual' (default), 'classic', 'banking', 'oldstyle' and 'fancy'
\moderncvcolor{orange}                               % color options 'black', 'blue' (default), 'burgundy', 'green', 'grey', 'orange', 'purple' and 'red'
%\renewcommand{\familydefault}{\sfdefault}         % to set the default font; use '\sfdefault' for the default sans serif font, '\rmdefault' for the default roman one, or any tex font name
%\nopagenumbers{}                                  % uncomment to suppress automatic page numbering for CVs longer than one page

% character encoding
%\usepackage[utf8]{inputenc}                       % if you are not using xelatex ou lualatex, replace by the encoding you are using
%\usepackage{CJKutf8}                              % if you need to use CJK to typeset your resume in Chinese, Japanese or Korean

% adjust the page margins
\usepackage[scale=0.9]{geometry}
%\setlength{\hintscolumnwidth}{3cm}                % if you want to change the width of the column with the dates
%\setlength{\makecvtitlenamewidth}{10cm}           % for the 'classic' style, if you want to force the width allocated to your name and avoid line breaks. be careful though, the length is normally calculated to avoid any overlap with your personal info; use this at your own typographical risks...

% personal data
\name{Tom}{Farley}
%\title{Resumé title}                               % optional, remove / comment the line if not wanted
\address{163 Pinnocks Way,}{Oxford, OX2 9DF, UK}{}% optional, remove / comment the line if not wanted; the "postcode city" and "country" arguments can be omitted or provided empty
\phone[mobile]{+44~75216~55595}                   % optional, remove / comment the line if not wanted; the optional "type" of the phone can be "mobile" (default), "fixed" or "fax"
%\phone[fixed]{+2~(345)~678~901}
%\phone[fax]{+3~(456)~789~012}
\email{tom.farley@ukaea.uk}                               % optional, remove / comment the line if not wanted
\homepage{www-users.york.ac.uk/\\$\sim$tpmf500}                         % optional, remove / comment the line if not wanted
\social[linkedin]{tom-farley-49021a98}                        % optional, remove / comment the line if not wanted
%\social[twitter]{jdoe}                             % optional, remove / comment the line if not wanted
\social[github]{github.com/TomFarley}                              % optional, remove / comment the line if not wanted
%\extrainfo{additional information}                 % optional, remove / comment the line if not wanted
\photo[70pt][0.4pt]{tfarley}                       % optional, remove / comment the line if not wanted; '64pt' is the height the picture must be resized to, 0.4pt is the thickness of the frame around it (put it to 0pt for no frame) and 'picture' is the name of the picture file
\quote{I am an experimental plasma physicist interested in the study of the tokamak scrape-of layer. I have experience analysing visible and infra-red camera data for the study of particle and energy transport respectively. I also have hands on experience working on small scale RF plasma experiments performing Langmuir probe measurements.}                                 % optional, remove / comment the line if not wanted

% bibliography adjustements (only useful if you make citations in your resume, or print a list of publications using BibTeX)
%   to show numerical labels in the bibliography (default is to show no labels)
\makeatletter\renewcommand*{\bibliographyitemlabel}{\@biblabel{\arabic{enumiv}}}\makeatother
%   to redefine the bibliography heading string ("Publications")
%\renewcommand{\refname}{Articles}

% bibliography with mutiple entries
%\usepackage{multibib}
%\newcites{book,misc}{{Books},{Others}}
%----------------------------------------------------------------------------------
%            content
%----------------------------------------------------------------------------------

\usepackage[disable]{todonotes}
\usepackage{mdwlist}		% use \begin{enumerate*} etc for compact lists
% \usepackage{pdfpages}
\setlength{\hintscolumnwidth}{3.8cm}  % 3.8cm, 4.2cm
%\reversemarginpar  % todonotes in left margin
%\setlength{\marginparwidth}{4cm}

\newcommand{\myname}[1]{{\underline{T. Farley}}}

\begin{document}
%\begin{CJK*}{UTF8}{gbsn}                          % to typeset your resume in Chinese using CJK
%-----       resume       ---------------------------------------------------------
\makecvtitle

\section{Education}
\subsection{PhD}
\cventry{2013--}{Fusion Centre for Doctoral Training (Fusion CDT)}{\newline University of Liverpool}{PhD student}{}
{\begin{itemize}%
\item Received a broad grounding in fusion research through taught courses at The University of York:% during first 6 months of PhD
\begin{itemize}
\item Courses included MCF and ICF relevant plasma physics, diagnostic techniques, data analysis, materials science, statistics, high performance computing and project management.
\item Will qualify for the FuseNet Certificate of Doctoral Training.
\end{itemize}
\end{itemize}}
\subsection{Integrated Masters}
\cventry{2009--2013}{Masters in Physics with Industrial Experience (F305)}{\newline The University of Bristol}{}{{First Class Honours}}  % arguments 3 to 6 can be left empty
{\begin{itemize}%
\item 12 month placement at the Culham Centre for Fusion Energy in IR thermography on JET.%resulting in a publication \cite{Arnoux2013b}.
\item Masters project in density functional theory resulted in a publication.
\item Received letters of commendation from the head of the School of Physics in 1\textsuperscript{st} and 2\textsuperscript{nd} years and a project commendation for my MSci project.
\end{itemize}}

\section{Research Experience}
\subsection{PhD}
\cventry{Jan 2016--}
{Visual camera measurements of filamentary transport in MAST}{\newline {Culham Centre for Fusion Energy}}{}{}
{\begin{itemize}%
\item Developed the Elzar suite of analysis tools for the identification, measurement and tracking of plasma filaments in fast camera data.
\item Applied pseudo Langmuir probe techniques to fast camera analysis for like with like comparison to large body of literature.
\item Joined trip to the HL-2A tokamak in Chengdu, China performing reciprocating probe measurements of the effects of fuelling on scrape-off layer profiles.
\item Performed experiential work for PPCF paper \cite{Militello2016a} and assisted in \cite{Walkden2017}.
\end{itemize}}

\cventry{Jan--March 2015}{Measurements of negative ion surface production from diamond materials}{\newline {PIIM Laboratory, Aix-Marseilles University, Marseilles, France}}{}{}
{\begin{itemize}%
\item Measured negative ions produced upon positive ion bombardment of diamond surfaces with a mass spectrometer and energy analyser. 
%\item Performed modelling of negative ion distribution functions using SRIM code.
\item Performed first of a kind measurements of monocrystaline diamond and extensively characterised temperature dependent properties of nanocrystaline diamond, contributing to publication \cite{Cartry2017}.
\end{itemize}}
\newpage
\cventry{Sep--Dec 2014}
{Characterisation of Small Negative Ion Facility}{\newline Culham Centre for Fusion Energy}
{}
{}{\begin{itemize}%
\item Commissioned the Langmuir probe and high resolution visible spectrometer diagnostics on the Small Negative Ion Facility (SNIF).
\item Identified and fixed RF earthing issues, improving performance of the plasma source.
\item Performed spectroscopic measurements of the ion source plasma composition, contributing to publication \cite{Zacks2017}.
\end{itemize}}

\cventry{March--Sep 2014}
{Langmuir probe and laser photo-detachment measurements of electronegative plasmas}
{\newline {University of Liverpool}}
{}
{}
{
\begin{itemize}%
\item Performed laser photo-detachment measurements of negative ion density in oxygen and hydrogen magnetron plasmas under various conditions.
\item Performed langmuir probe measurements of plasma density and temperature in a weakly magnetised plasma.
\end{itemize}
}

\subsection{Masters Project}
\cventry{Sep 2012--Sep 2013}
{DFT calculations of the superconducting properties of the \texorpdfstring{YIr\textsubscript{2}Si\textsubscript{2}}{YIr2Si2} polymorphs}
{\newline \emph{University of Bristol}}
{}
{}
{\begin{itemize}%
\item Performed Density Functional Theory (DFT) \textit{ab initio} calculations of the band structure and Fermi surface properties of the polymorphs of \texorpdfstring{YIr\textsubscript{2}Si\textsubscript{2}}{YIr2Si2}.
\item Predicted a superconducting transition temperature of $T_c= 2.58$K explained by intermediate-strength conventional electron-phonon coupling, resulting in a publication \cite{Billington2014}.
\end{itemize}}
\subsection{Undergraduate}
\cventry{Aug 2011--Aug 2012}
{Infra-red measurements of scrape-off layer power decay length in JET}
{\newline Culham Centre for Fusion Energy}{}{}
{\begin{itemize}%
\item Developed the tools required to measure the plasma scrape-off layer power decay length from infra-red images of the interior of the JET tokomak.
%\item Performed data analysis of large dedicated pulse database
%\item Learnt and became proficient in the IDL programing language
%\item Worked with large degree of independence with limited contact with supervisor
%\item Managed time effectively both to meet deadlines at work and cover material for university examinations though distance learning at home
%\item Maintained detailed log of work for discussing progress and challenges with supervisor
%\item Gained experience working in an international scientific research environment
%\item Produced novel results which are influencing the current ITER design review
\item Analysed a large dedicated pulse database, the results of which led to a publication in Nuclear Fusion \cite{Arnoux2013} and were presented by my supervisor at the 2012 IAEA conference \cite{Arnoux2012}.
\item Tools are now used by others and have initiated similar measurements on the COMPASS tokamak.
\end{itemize}}

% Publications from a BibTeX file without multibib
%  for numerical labels: \renewcommand{\bibliographyitemlabel}{\@biblabel{\arabic{enumiv}}}% CONSIDER MERGING WITH PREAMBLE PART
%  to redefine the heading string ("Publications"): \renewcommand{\refname}{Articles}
\nocite{*}
\bibliographystyle{cv_tfarley_custom_bib}   % cv_tfarley_custom_bib, unsrt
\bibliography{TFarley_publications3}                        % 'publications' is the name of a BibTeX file
% Publications from a BibTeX file using the multibib package
%\section{Publications}
%\nocitebook{book1,book2}
%\bibliographystylebook{plain}
%\bibliographybook{publications}                   % 'publications' is the name of a BibTeX file
%\nocitemisc{misc1,misc2,misc3}
%\bibliographystylemisc{plain}
%\bibliographymisc{publications}                   % 'publications' is the name of a BibTeX file

%\section{Publications}
%\cventry{year--year}{Job title}{Employer}{City}{}{}
%\cite{Zacks2017}
%\cite{Cartry2017}
%\cite{Walkden2016}
%\cite{Militello2016a}
%\cite{Billington2014}
%\cite{Arnoux2013b}
%\cite{Nunes2012}
%\cite{IAEA_synopsis}

\section{Conferences and Workshops}
\subsection{Oral}
\cventry{May 2017}
{Fusion Frontiers and Interfaces Workshop}{York}{\newline Fast Camera Analysis of Plasma Filaments}{}{}
\subsection{Posters}
\cventry{Oct 2017}
{59th Annual Meeting of the APS Division of Plasma Physics)}{\newline Milwaukee, Wisconsin, USA}{\newline Fast Camera Analysis of Plasma Filaments}{}
{}
\cventry{April 2017}
{44th IOP Plasma Physics Conference}{Oxford, UK}{\newline An Algorithm for the Analysis of Filaments in Fast Camera Data}{}
{}
\cventry{May 2016} 
{Fusion Frontiers and Interfaces Workshop}{York, UK}{\newline Pseudo Langmuir Probe Analysis  of Filaments in MAST Using Fast Cameras}{}
{}
\cventry{Nov 2014}
{FuseNet PhD Workshop}{Lisbon, Portugal}{\newline The SNIFF Caesium Free Negative Ion Source}{}
{}
\cventry{Oct 2014}
{4th International Symposium on Negative Ions, Beams and Sources}{IPP Garching, Germany}{\newline Caesium Free Negative Ion Sources}{}
{}
\cventry{May 2014}
{Fusion Frontiers and Interfaces Workshop}{York, UK}{\newline Laser Photo-detachment Measurements of Negative Ion Density}{}
{}

\section{Responsibilities}

\subsection{Groups}
\cventry{2017--}
{Software Developers Working Group}{Member of the Culham Software Developers Working Group (SDWG)}{}{}
{Site wide body responsible for coordinating resources for software developers.
}
\subsection{Meetings}
\cventry{2017--}
{Coding Discussion Group}{Founder and coordinator of the Culham Coding Discussion Group (CDG)}{}{}{Fortnightly meetings with online resources to share programming knowledge, expertise and resources.
}
\subsection{Supervision}
\cventry{4 months}
{Masters student}{\normalfont{Supervised masters student from University of Rome, Italy working on filament tracking for masters project}}{}{}{}
\cventry{3 months}
{Undergraduate project}{\normalfont{Supervised 3rd year undergraduate student from University of Cagliari, Italy working on filament detection for BSc project}}{}{}{}

\section{Outreach}
\cvitem{MAST-U tours}{Frequently lead MAST-U tours for visitors, open evening and open day attendees.}
\cvitem{GCSE work experience}{Supervised GCSE work experience student on placement at CCFE.}
\cvitem{Sun dome}{Helped with sun dome science workshop at local primary school.}

\section{Key Competencies}
%\cvlistitem{ Able to take initiative when presented with unfamiliar tasks}
%\cvlistitem{ Self-motivated and able to plan and structure work into clearly identified goals}
%\cvlistitem{ Ability to work under time pressure to meet deadlines }
%\cvlistitem{ Flexible and adaptable when working in a team environment}
%\cvlistitem{ Analytical ability to identify strengths and weaknesses of an argument or method}
%\cvlistitem{ Experience working on group projects involving remote collaboration and project management}
\cvlistitem{\textbf{Experimental experience}
\begin{itemize}
\item Experience working with r.f. plasma sources, compressed gas, vacuum systems and pumps and lasers.
\item Experience performing plasma measurements with langmuir probes, mass spectrometers and visible spectrometers.
%\item Experience developing analysis tools to process data
\end{itemize} }
\cvlistitem{\textbf{Programming experience}
\begin{itemize}
\item Experienced python programmer, OOP, TDD and HPC principals.
\item Experience with C, IDL and MATLAB.
\item Familiarity with C++, Bash scripting, Perl, Visual Basic and Fortran. 
%\item Familiar with high performance computing concepts and techniques. 
\item Experienced user of git version control and the the \LaTeX~typesetting language.
%\item Proficient user of MS office applications
\end{itemize} }
\cvlistitem{\textbf{Organisational skills}
\begin{itemize}
\item Excellent organisational skill, drawing up plans, maintaining detailed records and managing time effectively.
\end{itemize} }
\cvlistitem{\textbf{Communication skills}
\begin{itemize}
\item Communicate technical information in a competent and accessible manner, both verbally and in writing.
\item Perform well in a team, integrating readily into different teams and environments. 
\end{itemize}}
%\cvlistitem{ Regular user of linux with familiarity with shell scripting}
%\cvlistitem{Presentational skills to communicate technical information in a competent and accessible manner}


\section{Affiliations}
\cventry{2013--}
{\normalfont{Member of the \textbf{Institute of Physics}}}{}{}{}{}
\cventry{2017--}
{\normalfont{Member of the \textbf{American Physical Society}}}{}{}{}{}

\section{References}

\begin{cvcolumns}
\cvcolumn{Prof.James Bradley}{
Professor and Head of Group \\
Department of Electrical Engineering and Electronics\\
University of Liverpool \\
Liverpool, L69 3GJ, UK \\
\emailsymbol J.W.Bradley@liverpool.ac.uk \\
\fixedphonesymbol +44 (0)151 794 4545}

\cvcolumn{Dr James Harrison}{
Deputy Head of Tokamak Science \\
Culham Centre for Fusion Energy \\
Office: D3/2.04	  \\
Culham Science Centre \\
Abingdon, OX14 3DB, UK \\
\emailsymbol James.Harrison@ukaea.uk \\
\fixedphonesymbol +44(0) 1235 46 6209}
\end{cvcolumns}

\begin{cvcolumns}
\cvcolumn{Professor Antony Carrington}{
Personal tutor \\
HH Wills Physics Laboratory \\
University of Bristol \\
Tyndall Avenue \\
Bristol \\
BS8 1TL \\
\emailsymbol A.Carrington@bristol.ac.uk \\
\fixedphonesymbol +44 (0)117 928 8713}

%\cvcolumn{Dr Fulvio Militello}
%  {Head of the Exhaust Topic Area \\
%Culham Centre for Fusion Energy \\
%Office: D3/1.76	  \\
%Culham Science Centre \\
%Abingdon, OX14 3DB, UK \\
%\emailsymbol fulvio.militello@ukaea.ac.uk \\
%\fixedphonesymbol +44(0) 1235 46 6253}
\end{cvcolumns}
\todo{Update webpage}

%\begin{cvcolumns}
%\cvcolumn{Dr James Harrison}{
%Deputy Head of Tokamak Science \\
%Culham Centre for Fusion Energy \\
%Office: D3/2.04	  \\
%Culham Science Centre \\
%Abingdon, OX14 3DB, UK \\
%\emailsymbol James.Harrison@ukaea.uk \\
%\fixedphonesymbol +44(0) 1235 46 6209}
%  
%\cvcolumn{Dr Gilles Arnoux}
%  {
%Section Leader: \\Viewing Systems and Thermal Measurements \\
%Plasma Boundary Group, JET Diagnostics Department \\
%Culham Centre for Fusion Energy \\
%Bldg/Office: K1/1/14 \\
%Culham Science Centre \\
%Abingdon, OX14 3DB, UK \\
%\emailsymbol gilles.arnoux@ccfe.ac.uk \\
%\fixedphonesymbol +44(0) 1235 46 4809}
%\end{cvcolumns}

\clearpage
%-----       letter       ---------------------------------------------------------
% recipient data
\recipient{UKAEA recruitment team}{Culham Centre for Fusion Energy\\Culham Science Centre \\Abingdon \\OX14 3EB}
\date{\today}
\opening{\textbf{Application for position of \textit{Scrape-Off Layer Turbulence Physicist}}\\~\\
Dear Sir or Madam,}
\closing{Yours faithfully,}
\enclosure[Attached]{curriculum vit\ae{}}          % use an optional argument to use a string other than "Enclosure", or redefine \enclname
\makelettertitle

I should like to apply for the role of \textit{Scrape-Off Layer Turbulence Physicist}, for which I feel I am particularly well suited.
% and I hope to demonstrate why.
I see MAST-U, with its unique visible imaging capabilities, cutting edge fast framing cameras, versatile divertor science facility and powerful mid-plane reciprocating probe system, as an extremely exciting opportunity for significantly advancing our understanding of scrape-off layer (SOL) physics.
In particular, it will be fascinating to investigate the effects of the super-X divertor and varied divertor and detachment regimes on both upstream and divertor SOL dynamics.

My PhD project as part of the Fusion CDT, entitled `\textit{Analysis of Scrape-Off Layer Plasma Filaments Through Fast Visible Imaging}', has given me a strong background in SOL physics, particularly filamentary transport of particles.
I already have considerable experience in the analysis of fast visible camera data on MAST and I see this position as the perfect opportunity to develop this expertise further, particularly in taking advantage of the stereoscopic imaging measurements that will be possible on MAST-U.
My existing familiarity with the MAST-U tokamak and its visual camera and Langmuir probe systems will enable me to quickly start applying MAST-U's diagnostics to their full potential.

My work performing novel SOL power decay length measurements, using IR thermography on JET, further demonstrates my experience with SOL physics, imaging measurements and working within different research groups.
My work on neutral beam negative ion sources has provided me with valuable experience working hands on with vacuum systems and performing Langmuir probe measurements in weakly magnetised RF plasmas.
I am highly proficient in python and have developed the large, sophisticated Elzar suite of tools for enhancing, analysing and reviewing fast camera data.

I am very well organised and believe I will operate the diagnostic systems with efficiency, applying good practice and maintaining accurate records of measurements, adaptations and changes.
I appreciate my experience with electrical probes is not at the same level as that which I have with imaging techniques, but I believe I can rapidly develop these skills.
My strong record of publications at this early stage in my career, with contributions towards 8 publications and a first author paper in preparation, demonstrate my proven ability to conduct high quality scientific research.

While my anticipated thesis submission date is not until early 2019, I hope my strengths and experience in the area will justify a later start date for the position.
I attach my CV with further evidence of my suitability for this position and I would be grateful for the opportunity to demonstrate my capabilities further at interview. 
%I look forward to hearing from you.

\makeletterclosing

\newpage
\textbf{Describe a situation where you have had to deliver something, be it a project or service, to a high standard. How did you know it reached the standard required? Is there anything you could have done to improve what you delivered? (Please provide examples where appropriate)	}

At the commencement of my current PhD project, a deadline within the group was fast approaching for an Enabling Research project focusing on the benchmarking of a number of plasma turbulence codes. I was tasked with delivering the experimental camera measurements of scrape-off layer filaments with which to initialise the codes and compare their output. I thus had to get up to speed quickly with the project and develop the required analysis tools. I made sure to get clear descriptions of the requirements so as to ensure I delivered the required results. Regular checks with colleagues ensured the measurements met the standards requested, which were ultimately confirmed by the peer review process with which the resulting paper was assessed. This experience taught me how I could improve the recording of data and the settings used to produce them, for the purposes of data provenance and guiding subsequent work.


\textbf{Describe a time when you have worked well and achieved success with a group of others. What was the outcome? What was your contribution? (Please provide examples where appropriate) }

During training as part of the Fusion CDT course I undertook a group project with 4 others to design a hypothetical diagnostic which was not implemented on an existing or planned tokamak. We were located geographically across 4 universities and so had to plan, communicate, share resources and combine our output efficiently using online tools.  I took a proactive role in scheduling our meetings, maintaining good communication and keeping the group to schedule. In the early stages of the project I reviewed possible diagnostic ideas and compared my findings with the other team members.  From a shortlist, we decided the most feasible and effective option was a laser-induced breakdown spectroscopy diagnostic to study erosion, deposition and fuel retention in the ITER divertor. I produced a document on Google Drive summarising the physics case, technical requirements and key unsolved questions relating to the project, which the team used to collate our findings and update information as new literature was discovered. This enabled us to work as a cohesive team combining each individual's contribution effectively. We broke up the requirements for the project amongst the group and I took responsibility for the project management plan, formulating the work breakdown structure. This required learning to use new software and techniques and maintaining close contact with my fellow team members, in order to coordinate component lead times and personnel effort to ensure the project schedule, critical path and costs would meet externally imposed ITER requirements. The project was assessed through a report and presentation to which we each contributed a section. The project was awarded a distinction, with the highest marks in the project awarded to the management plan that I worked on. This section was described in feedback as ``extremely     well     thought     through     and     presented''.

\textbf{Describe one of your biggest challenges where you had to persevere to succeed. What happened and what did you learn from this? (Please provide examples where appropriate)} 

During my PhD I collaborated with a group at PIIM Laboratory at Aix-Marseilles University, measuring negative ion surface production from diamond materials. I had been allocated 10 weeks of experimental time on a specialised mass-spectroscopy experiment, but on arrival was informed there was a backlog of users requiring time on the apparatus. As a result, it was not until 6 weeks into my visit that I had the planned exclusive access to the apparatus in order to perform my measurements. This led to a great deal of time pressure to achieve my goals. I thus had to rapidly change my plans and I learnt to apply the risk mitigation strategies developed in my grant proposal to achieve the best outcome. 
During this initial period I took part in the other user's experiments to gain as much experience with the apparatus as possible before starting my measurements, assigned contingency time and switched to prioritising one of the two planned sets of measurements. The initial period was also used to perform modelling work to better inform the experiments. I formed detailed, structured plans and sought advice from others to ensure my experimental time was used as efficiently as possible. Despite being a hectic and stressful period, through effective planning, time management and perseverance, I succeeded in meeting my key goals. These included fully characterising the negative ion surface production properties of nanocrystalline diamond materials for the first time, eliciting interesting high temperature behaviour. The project had many valuable outcomes and led in part to a journal publication.

\textbf{Please describe your motivation for working at UKAEA.}

I see climate change and global energy security as two of the most pressing issues of our time and the pursuit of fusion energy as a vital endeavour in ensuring the best possible future for life on Earth. I want to work for UKAEA as it is home to world leading fusion expertise and two flagship fusion experiments. UKAEA is therefore the place I believe I can contribute the most to fusion research. I believe working for UKAEA would be highly rewarding as the organisation's future achievements will have the potential to benefit generations to come. At MAST-U in particular, the unique capabilities of its fast visible cameras, divertor science facility and super-X divertor present an exciting opportunity to make unique and challenging measurements, which will better inform our understanding of scape-off layer transport. I believe my experience, expertise and skills make me ideally suited to make an important contribution to this field.

%I see global warming and global energy security as two of the most pressing issues of our time and the pursuit of fusion energy as a vital endeavour in ensuring the best possible future for us all. I see Culham with its two flagship fusion experiments and world leading expertise as the best place for me to contribute to this most valuable and rewarding of undertakings, with the potential to benefit many generations to come. At MAST-U in particular the unique capabilities of its fast visible cameras, divertor science facility and super-X divertor present an exciting opportunity to make unique and challenging measurements to better inform our understanding of scape-off layer transport.

%\listoftodos

%\clearpage\end{CJK*}                              % if you are typesetting your resume in Chinese using CJK; the \clearpage is required for fancyhdr to work correctly with CJK, though it kills the page numbering by making \lastpage undefined
\end{document}


%% end of file `template.tex'.
